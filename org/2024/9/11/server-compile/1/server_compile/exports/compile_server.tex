% Created 2024-09-11 三 13:45
% Intended LaTeX compiler: xelatex
\documentclass[cn,10pt]{elegantbook}



\extrainfo{Victory won\rq t come to us unless we go to it. --- M. Moore}
\setcounter{tocdepth}{3}
\logo{logo.png}
\cover{theme.jpg}


\usepackage{fontspec}
\renewcommand{\familydefault}{\sfdefault}
\usepackage{minted}
\newminted{cpp}{frame=single,linenos,breaklines}
\newminted{java}{frame=single,linenos,,breaklines}
\newminted{shell}{frame=single,linenos,breaklines}
\newminted{ruby}{frame=single,linenos,breaklines}
\newminted{typescript}{frame=single,linenos,breaklines}
\newminted{js}{frame=single,linenos,breaklines}
\newminted{sql}{frame=single,linenos,breaklines}
\newminted{common-lisp}{frame=single,linenos,breaklines}
\newminted{lisp}{frame=single,linenos,breaklines}
\newminted{yaml}{frame=single,linenos,breaklines}
\newminted{xml}{frame=single,linenos,breaklines}
\newminted{tex}{frame=single,linenos,breaklines}
\newminted{rust}{frame=single,linenos,breaklines}
\newminted{python}{frame=single,linenos,breaklines}
\newminted{html}{frame=single,linenos,breaklines}
\newminted{groovy}{frame=single,linenos,breaklines}
\newminted{go}{frame=single,linenos,breaklines}
\newminted{c++}{frame=single,linenos,breaklines}
\newminted{cmake}{frame=single,linenos,breaklines}
\newminted{make}{frame=single,linenos,breaklines}
\newminted{abap}{frame=single,linenos,breaklines}
\author{郭隆基}
\date{\today}
\title{服务器内核编译}
\hypersetup{
 pdfauthor={郭隆基},
 pdftitle={服务器内核编译},
 pdfkeywords={},
 pdfsubject={},
 pdfcreator={Emacs 29.3.50 (Org mode 9.6.15)}, 
 pdflang={Zh}}
\begin{document}

\maketitle
\frontmatter
\tableofcontents
\mainmatter


\chapter{5.10 rpm 内核包编译步骤}
\label{sec:org74e23b6}

(1 clone \url{https://gerrit-dev.uniontech.com/admin/repos/Kernel-rpmbuild} 这个项目。将其拷贝到99 编译服务器。

(2 clone \url{https://gerrit-dev.uniontech.com/admin/repos/kernel-5.10-server} 项目代码。使用

\begin{verbatim}
git archive --prefix=linux-5.10.0-74.4.1/ -o linux-5.10.0-74.4.1.tar --format=tar branch_name
xz -z linux-5.10.0-74.4.1.tar
\end{verbatim}

做好源码包,注意一定要把版本号写对,一定要代码包上一层prefix目录。5.10 使用 tar.xz 包。

(3 将包上传到服务器的 Kernel-rpmbuild 中的 rpmbuild-5.10/SOURCES 目录。

(4 修改 rpmbuild-5.10/SPECS/kernel.spec

\begin{verbatim}
diff --git a/rpmbuild-5.10/SPECS/kernel.spec b/rpmbuild-5.10/SPECS/kernel.spec
index af53c46c5..8721e8491 100644
--- a/rpmbuild-5.10/SPECS/kernel.spec
+++ b/rpmbuild-5.10/SPECS/kernel.spec
@@ -28,8 +28,8 @@
 # define buildid .local

 %define rpmversion 5.10.0
-%define pkgrelease 74
-%define gittagid 5.10.0-74
+%define pkgrelease 74.4.1
+%define gittagid 5.10.0-74.4.1

 # allow pkg_release to have configurable %%{?dist} tag
 %define specrelease %{pkgrelease}%{?dist}
@@ -1029,20 +1029,20 @@ BuildKernel() {
       cp arch/$Arch/boot/zImage.stub $RPM_BUILD_ROOT/lib/modules/$KernelVer/zImage.stub-$KernelVer || :
     fi

-    %if %{signkernel}
-    KernelExtension=${KernelImage##*.}
-    if [ "$KernelExtension" == "gz" ]; then
-        SignImage=${KernelImage%.*}
-    else
-        SignImage=$KernelImage
-    fi
-    sbsign --hwkey 2 --cert %{SOURCE14} --output vmlinuz.signed $SignImage
-    if [ ! -s vmlinuz.signed ]; then
-        echo "sbsign failed"
-        exit 1
-    fi
-    mv vmlinuz.signed $KernelImage
-    %endif
+    #%if %{signkernel}
+    #KernelExtension=${KernelImage##*.}
+    #if [ "$KernelExtension" == "gz" ]; then
+    #    SignImage=${KernelImage%.*}
+    #else
+    #    SignImage=$KernelImage
+    #fi
+    #sbsign --hwkey 2 --cert %{SOURCE14} --output vmlinuz.signed $SignImage
+    #if [ ! -s vmlinuz.signed ]; then
+    #    echo "sbsign failed"
+    #    exit 1
+    #fi
+    #mv vmlinuz.signed $KernelImage
+    #%endif

     $CopyKernel $KernelImage \
                 $RPM_BUILD_ROOT/%{image_install_path}/$InstallName-$KernelVer
@@ -2065,6 +2065,9 @@ fi
 #
 #
 %changelog
+* Tue Sep 10 2024 Longji Guo <guolongji@uniontech.com> - 5.10.0-74.4.1
+- uos: Revert "anolis: Revert "ext4: fix bad checksum after online resize" [T286689]" [G006383]
+
 * Mon Apr 22 2024 Li Hongbin <lihongbin@uniontech.com> - 5.10.0-74

 * Wed Apr 10 2024 caina <caina@uniontech.com> - 5.10.0-73
\end{verbatim}

(5 将源码的配置文件拷贝至内核打包仓对应目录的SOURCES文件下,并命名为 \textbf{kernel-\$aarch.config} 、
*kernel-\$aarch-debug.config*。419E内核该步骤可省略。

我将源码中的 uos510\_defconfig 拷贝到服务器的 rpmbuild-5.10/SOURCES/kernel-5.10.0-x86\_64.config 和
rpmbuild-5.10/SOURCES/kernel-5.10.0-x86\_64-debug.config

修改abi 文件版本,如下:
\begin{verbatim}
尚未暂存以备提交的变更:
  (使用 "git add/rm <文件>..." 更新要提交的内容)
  (使用 "git restore <文件>..." 丢弃工作区的改动)
        删除:     kernel-abi-whitelists-5.10.0-46.tar.bz2
        删除:     kernel-kabi-dw-5.10.0-46.tar.bz2

未跟踪的文件:
  (使用 "git add <文件>..." 以包含要提交的内容)
        kernel-abi-whitelists-5.10.0-74.4.1.tar.bz2
        kernel-kabi-dw-5.10.0-74.4.1.tar.bz2
\end{verbatim}

(6 在 rpmbuild-5.10/SPECS/ 目录下执行编译命令

编译 e 版(uel 后缀):time rpmbuild --define "\_topdir /home/glj/Kernel-rpmbuild/rpmbuild-5.10" --define "dist uel20" -ba kernel.spec

编译 a 版 (uelc 后缀):time rpmbuild --define "\_topdir /home/glj/Kernel-rpmbuild/rpmbuild-5.10" --define  -ba kernel.spec

最终编译好的包在 rpmbuild-5.10/RPMS/x86\_64/ 当中。

\chapter{签名}
\label{sec:org21876bb}

将上面生成的 rpm 包上传到签名服务器,后执行:

find . -name '*.rpm' | xargs -i rpmsign --addsign \{\}

现在就完成了。
\end{document}

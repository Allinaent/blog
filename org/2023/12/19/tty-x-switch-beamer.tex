\documentclass[8pt,aspectratio=43,mathserif,table]{beamer}
% 设置为 Beamer 文档类型,设置字体为 10pt,长宽比为16:9,数学字体为 serif 风格
\batchmode

\usepackage{graphicx}
\usepackage{animate}
\usepackage{hyperref}

% 导入一些用到的宏包
\usepackage{amsmath,bm,amsfonts,amssymb,enumerate,epsfig,bbm,calc,color,ifthen,capt-of,multimedia,hyperref}
\usepackage{xeCJK} %导入中文包
\setCJKmainfont{SimHei} %字体采用黑体  Microsoft YaHei

\usetheme{Berlin} %主题
\usecolortheme{sustech} %主题颜色

\usepackage[ruled,linesnumbered]{algorithm2e}

\usepackage{fancybox}
\usepackage{xcolor}
\usepackage{times}
\usepackage{listings}

\usepackage{booktabs}
\usepackage{colortbl}

\newcommand{\Console}{Console}


%\setsansfont{Microsoft YaHei}
%\setmainfont{Microsoft YaHei}

\definecolor{mygreen}{rgb}{0,0.6,0}
\definecolor{mymauve}{rgb}{0.58,0,0.82}
\definecolor{mygray}{gray}{.9}
\definecolor{mypink}{rgb}{.99,.91,.95}
\definecolor{mycyan}{cmyk}{.3,0,0,0}

\lstset{ %
  basicstyle=\tiny
  backgroundcolor=\color{white},   % choose the background color
  basicstyle=\footnotesize\rmfamily,     % size of fonts used for the code
  columns=fullflexible,
  breaklines=true,                 % automatic line breaking only at whitespace
  captionpos=b,                    % sets the caption-position to bottom
  tabsize=4,
  commentstyle=\color{mygreen},    % comment style
  escapeinside={\%*}{*)},          % if you want to add LaTeX within your code
  keywordstyle=\color{blue},       % keyword style
  stringstyle=\color{mymauve}\ttfamily,     % string literal style
  % numbers=left, 
  % frame=single,
  rulesepcolor=\color{red!20!green!20!blue!20},
  % identifierstyle=\color{red},
  language=c
}


% 题目,作者,学校,日期
\title{ 闪屏花屏问题分析}
\subtitle{\fontsize{9pt}{14pt}\textbf{ 内核与用户态显示切换用户涉及的知识点}}
\author{ 演讲人:郭隆基  \newline \newline }
\institute{\fontsize{5pt}{14pt} 统信软件技术有限公司}
\date{\today}

\tolerance=1
\emergencystretch=\maxdimen
\hyphenpenalty=10000
\hbadness=10000

% 学校Logo
% \pgfdeclareimage[height=0.5cm]{sustech-logo}{sustech-logo.pdf}
% \logo{\pgfuseimage{sustech-logo}\hspace*{0.3cm}}

\AtBeginSection[]
{
  \begin{frame}<beamer>
    \frametitle{\textbf{目录}}
    \tableofcontents[currentsection]
  \end{frame}
}
\beamerdefaultoverlayspecification{<+->}
% -----------------------------------------------------------------------------
\begin{document}
% -----------------------------------------------------------------------------

\frame{\titlepage}

\section[目录]{}   %目录
\begin{frame}{目录}
  \frametitle{目录}
  \tableofcontents
\end{frame}

% -----------------------------------------------------------------------------
\section{引言}  %引言
\subsection{问题背景}
\begin{frame}{问题背景}
  \begin{columns}[T] % align columns
    \begin{column}<0->{.30\textwidth}
      \begin{figure}[thpb]
        \centering
        \resizebox{1\linewidth}{!}{
          \includegraphics{figures/question.pdf}
        }
        % \includegraphics[scale=1.0]{figurefile}
        % \caption{ 此处是图片的名称}
        \label{fig:campus}
      \end{figure}
    \end{column}%
    \hfill%
    \begin{column}<0->{.75\textwidth}
      \begin{itemize}
      \item<1-> 切换用户、锁屏界面登陆系统时,使用Raedon Oland 显卡机型性能劣化
        \begin{itemize}
        \item<1-> 为什么切换用户的时候有概率会黑一下屏?
        \end{itemize}
      \item<2-> 多用户切换测试高概率闪屏
        \begin{itemize}
        \item<2-> 和上面是同一个问题,极快的闪屏没有找到问题可以分析的路径
        \end{itemize}
      \end{itemize}
    \end{column}%
  \end{columns}
\end{frame}

\subsection{分析思路}
\begin{frame}{分析思路}
  分析闪屏问题的一般思路
  \begin{block}{具体步骤}
    \begin{itemize}
    \item<0->  扫盲;了解内核和应用层相关的模块
    \item<0->  猜;跟据了解的知识猜测可能的方向,不一定准确。现象多而且复杂一定要
      做排除,很多现象是干扰,甚至和调式环境有关。
    \item<0->  复现;找多种复现方式和手段,找到出问题与不出问题的区别
    \item<0->  分析;方向准确后分析内核和应用层模块代码逻辑。也用一些手段做验证
    \end{itemize}
  \end{block}
\end{frame}

\subsection{调试手段}
\begin{frame}{调试手段}
  %\frametitle{调式手段}
  \begin{figure}[!t]
    \centering
    \includegraphics[width=2in]{figures/debug.pdf}
    \caption{bpftrace 真好用}
    \label{figure3_debug}
  \end{figure}
  \begin{center}
    在分析我这个问题时我用的调试方法有 bpftrace gdb strace  \\
  \end{center}
    熟练应用 bpftrace 编程和gdb 的调试是解决问题提速的加分项,要学会写bpfrace 程序。
    如果是主要看内核中的流程,那qemu 也能提高效率。
\end{frame}


\section{误入迷途}  %自我介绍
\subsection{ 举例}
\begin{frame}{Edid}
  在 Oland 的显卡中有一个 VGA 循环读取 Edid 检测VGA 是否有插拔的内核进程,一度
  以为是用户切换调用 drm\_ctrc\_set\_config 也读取edid ,两个读取edid 冲突时就
  会造成黑屏。
  证伪的方法很简单,把这个循环时间调长。从10s 调到3600s ,发现问题仍然存在。
  之前的分析在:\\ https://guolongji.xyz/post/tty-x-switch/
\end{frame}

\begin{frame}{ 其它干扰}
  \begin{block}{ 举例}
    \begin{itemize}
    \item<0->  在drm\_do\_probe\_dde\_edid 调用栈的函数中加一些打印,会影响问题
      复现的概率。
    \item<0->  安装相关的 gdb 调式包会影响问题复现的概率,有时会让问题完全不复现。
    \item<0->  因为切入,切出的 xorg 都会收到 SIGUSR1 信号。用 {\tiny kill -SIGUSR1 pid} 来模
      拟。这样测试有时也会黑屏,但与问题关系不大。
    \end{itemize}
  \end{block}
\end{frame}

\subsection{总结}
\begin{frame}{总结}
  \begin{block}{总结}
    EEPROM 的读速度
    \begin{itemize}
    \item<0-> EEPROM 的读取速度和 FLASH 、SRAM 相比虽然慢一点,但是也差不了多少,只
      是写的速度很慢。
    \end{itemize}

    两个读取都是通过DDC 读 EDID
    \begin{itemize}
    \item<0-> VGA 的{\tiny INIT\_DELAYED\_WORK(&dev->mode\_config.output\_poll\_work, output\_poll\_execute);}
      可以修改 {\tiny \#define DRM\_OUTPUT\_POLL\_PERIOD (10*HZ)} 将10 改成 3600 。
    \item<0-> 另一个是通过{\tiny drm\_do\_probe\_dde\_edid} 读取。
    \item<0-> 因为同步导致的花屏闪屏,调整上下文的时序,增加延时会导致现象的不同。
    \end{itemize}
    例子中的干扰项有一点迷惑性,要排除这些干扰需要经验的积累。
  \end{block}
\end{frame}

\section{切中要害}
\subsection{bpftrace 分析关键函数}
\begin{frame}{bpftrace 分析关键函数}
  使用{\tiny sudo stackcount-bpfcc drm\_crtc\_helper\_set\_config} 多次测试,使用 bpf 分析
  调用,发现出问题的时候就出现下面的调用栈:
  \begin{block}{调用栈}
    {\tiny
      drm\_crtc\_helper\_set\_config \\
      amdgpu\_display\_crtc\_set\_config \\
      \_\_drm\_mode\_set\_config\_internal \\
      drm\_crtc\_force\_disable \\
      drm\_framebuffer\_remove \\
      drm\_mode\_rmfb\_work\_fn \\
      process\_one\_work \\
      worker\_thread \\
      kthread \\
      ret\_from\_fork \\
      \ \ kworker/4:3 [8160] \\
    }
  \end{block}
  \begin{center}
    \textcolor{mymauve}{ 问题很容易复现,关键就是解释这个逻辑的过程}
  \end{center}
\end{frame}

\subsection{ 逻辑分析}
\begin{frame}{ 一句话}
  用一句话来解释的话就是 setcrtc 是异步的,而Oland 显卡执行这个的时间不确定。
  Oland 显卡使用的 {\tiny DRM\_IOCTL\_MODE\_RMFB} 这个ioctl不是原子性的,drm\_framebuffer\_remove 调用的legacy\_remove\_fb也不是
  原子性的,所以无法失败回退。不能重新调用执行正确的逻辑怎么办?可以为内核增加容错的逻辑用一种有不太好的方式掩盖更大的问题。 \\
  ~\\

  xorg 收到信号后会调用 setcrtc 的ioctl,整体执行的时间比较长且此函数异步,下面又调用 rmfb 的ioctl,导致 sigusr1 处于pending
  状态的时间也比较长。rmfb 的ioctl如要使用 drm\_framebuffer\_remove 的话,会返回 -EINTR 。xorg 没有重新调用
  rmfb 这个ioctl 的逻辑那就不用这个函数,另外写一个处理过程,再内核空间做完 rmfb 的动作。

  不支持原子setcrtc 的旧架构的显卡,可能读写寄存器有时快一点有时慢一点,就会有时导致  drm\_framebuffer\_read\_refcount(fb) > 1
  在这种情况下就不是调用 drm\_framebuffer\_put(fb) 做释放。而是
  让内核在内核空间的共享任务队列当中创建一个任务,以解决可能的逻辑问题。这个函数执行的时间比
  较长,就会黑屏。\\
  ~\\

  具体细节在drm\_mode\_rmfb这个函数当中。这个问题是机制的问题,并不好解决。
\end{frame}

\begin{frame}{ xorg 断点一}
  下面是切出的 xorg 收到信号和 drmModeSetCrtc 断点的堆栈。切出的时候会调用 setcrtc 的
  ioctl 一次。会调用三次rmfb (和创建的有户数相等)。所以是切出的时候导致的黑屏。
  \begin{figure}[thpb]
        \centering
        \resizebox{1\linewidth}{!}{
          \includegraphics[width=6in]{figures/gdb1.pdf}
        }
        % \includegraphics[scale=1.0]{figurefile}
        % \caption{ 此处是图片的名称}
        \label{fig:campus}
  %\begin{figure}[!t]
  %  \centering
  %  \includegraphics[width=5in]{figures/gdb1.pdf}
  %  \caption{退出的xorg}
  %  \label{gdb1}
  \end{figure}
\end{frame}

\begin{frame}{ xorg 断点二}
  下面是切入的 xorg 收到信号和 drmModeSetCrtc 断点的堆栈,有黑屏切入的时候会调用setcrtc 的
  ioctl 两次,不会调用rmfb 。
  \begin{figure}[thpb]
        \centering
        \resizebox{1\linewidth}{!}{
          \includegraphics[width=6in]{figures/gdb2.pdf}
        }
        % \includegraphics[scale=1.0]{figurefile}
        % \caption{ 此处是图片的名称}
        \label{fig:campus}
  % flip
  \end{figure}
\end{frame}

\begin{frame}{ 内核逻辑}
  这里有一个比较绕的 fbs 参数传递的地方。
  \lstinputlisting[
  basicstyle=\tiny,
  %lastline=11,
  language=C,
  frame=single,
  %caption=First ten lines of some Python code,
  label=c]
  {demo1.c}

\end{frame}

\begin{frame}{ 内核逻辑}

  用户态每open 一个 fb 并创建一个fd 的时候,内核都会创建一个 drm\_file 用于关联
  用户空间的 fd 。drm\_file 有一个成员是 fbs ,用于存储与用户态的fd 相关联的全
  部fb 。
  \lstinputlisting[
  basicstyle=\tiny,
  %lastline=11,
  language=C,
  frame=single,
  %caption=First ten lines of some Python code,
  label=c]
  {demo2.c}

\end{frame}

\begin{frame}{ 内核逻辑}
  直接调用 drm\_framebuffer\_remove 去删除fb 的话,会被信号打断。
  \lstinputlisting[
  basicstyle=\tiny,
  %lastline=11,
  language=C,
  frame=single,
  %caption=First ten lines of some Python code,
  label=c]
  {demo3.c}

\end{frame}

\begin{frame}{ 内核逻辑}
  对 crtc 和 plane 这些资源做释放。
  \lstinputlisting[
  basicstyle=\tiny,
  %lastline=11,
  language=C,
  frame=single,
  %caption=First ten lines of some Python code,
  label=c]
  {demo4.c}

\end{frame}

\begin{frame}{ 内核逻辑}
  setcrtc 的ioctl 过程当中会增加 fb 的refcount 。
  \lstinputlisting[
  basicstyle=\tiny,
  %lastline=11,
  language=C,
  frame=single,
  %caption=First ten lines of some Python code,
  label=c]
  {demo5.c}

\end{frame}

\begin{frame}{ 内核逻辑}
  在 drm 框架当中,framebuffer 的引用计数在 drm\_framebuffer\_lookup 的时候会用,
  在用户空间open 的时候会用,还有在内核drm\_client 做渲染的时候也会用。drm\_mode\_rmfb
  的时候也要引用计数变为零之后才会释放。为了防止内核或系统卡死,framebuffer refcount 的处
  理要比较小心。
\end{frame}


\section{ ioctl的过程}
\begin{frame}{ 切出的 xorg IOCTL 过程}
  % {\tiny strace -e ioctl -p 899}
  切出会调用rmfb ioctl 三次(和创建的用户数相等),不论是否黑屏。
  \begin{figure}[thpb]
        \centering
        \resizebox{0.7\linewidth}{!}{
          \includegraphics[width=0.5\textwidth]{figures/ioctl.pdf}
        }
        % \includegraphics[scale=1.0]{figurefile}
        % \caption{ 此处是图片的名称}
        \label{fig:campus}
  % flip
  \end{figure}
\end{frame}

\begin{frame}{ 切入的 xorg IOCTL 过程}
  %{\tiny strace -e ioctl -p 899}
  \begin{figure}[thpb]
        \centering
        \resizebox{0.7\linewidth}{!}{
          \includegraphics[width=2in]{figures/ioctl2.pdf}
        }
        % \includegraphics[scale=1.0]{figurefile}
        % \caption{ 此处是图片的名称}
        \label{fig:campus}
  % flip
  \end{figure}
\end{frame}

\begin{frame}{ 交互的过程}
  \begin{block}{\textbf{ 流程}}
     \begin{itemize}
     \item<0-> 上层应用切换用户一是用户按下 Ctrl+Alt+F1 到 F12 快捷键,触发键盘中断
       ;二是通过 dde-session-shell (switchToUser) - lightdm - systemd-logind (用户会话与tty 关联)
  \item<0-> 切换用户的本质是资源的切换,主要是 systemd-logind 分配 cgroup 的切换和 tty 的切换。
  \item<0-> tty 驱动程序检查输入的按键是否是 VT 切换快捷键。如果是,则调用 vt\_ioctl() 函数进行 VT 切换。
  \item<0-> vt\_ioctl() 函数将 VT 切换请求转发给 console 驱动程序进行处理。
  \item<0-> console 驱动程序根据请求切换到对应的 VT,并更新对应的 tty 和终端设备状态。
  \item<0-> 一旦 VT 切换完成,console 驱动程序调用 vt\_event() 函数向用户空间发送 VT 切换事件。
  \item<0-> Xorg 服务器接收到 VT 切换事件,并调用 xf86VTEnter() 或 xf86VTLeave() 函数来响应事件。
  \item<0-> xf86VTEnter() 或 xf86VTLeave() 函数中,Xorg 服务器会暂停或恢复相关的图形设备、输入设备和输出设备,
    并执行相应的回调函数。
    \end{itemize}
  \end{block}
\end{frame}

\begin{frame}{ 交互的过程}
  % 用 tikz 来画一张流程图,这个不难
  \begin{figure}[thpb]
        \centering
        \resizebox{0.7\linewidth}{!}{
          \includegraphics[width=2in]{figures/tty-x-switch.png}
        }
        % \includegraphics[scale=1.0]{figurefile}
        % \caption{ 此处是图片的名称}
        \label{fig:campus}
  % flip
  \end{figure}
\end{frame}


% 要增加两个图来把这部分梳理一下,一个图是信号的;另一个图是内核处理部分。
\subsection{ 信号的处理流程}
\begin{frame}
  \frametitle{ 关于 sigusr1 信号}
  \begin{block}{\textbf{ 即是切出信号也是切入信号}}
    \begin{itemize}
    \item<0-> 点击切换按钮会调用vt\_ioctl 的VT\_ACTIVATE 和VT\_RELDISP 。
    \item<0-> 首先调用 VT\_ACTIVATE 的ioctl 用于激活目标虚拟终端,过程是主要在vt 驱动中:
      set\_console - schedule\_console\_callback - console\_work -  console\_callback - change\_console
      vt 切换的过程1.内核会向切出的 xorg 发 sigusr1 信号通知它要做处理。
    \item<0-> 然后调用VT\_RELDISP 的ioctl 用于释放当前终端。
      complete\_change\_console - kill\_pid(vc->vt\_pid, vc->vt\_mode.acqsig, 1)
      vt 切换的过程2.切出的xorg 向切入的xorg 发sigusr1 信号。
    \end{itemize}
  \end{block}
\end{frame}

\begin{frame}{ 交互的过程}
  % 用 tikz 来画一张流程图,这个不难
  \begin{figure}[thpb]
        \centering
        \resizebox{0.7\linewidth}{!}{
          \includegraphics[width=2in]{figures/sigusr1.png}
        }
        % \includegraphics[scale=1.0]{figurefile}
        % \caption{ 此处是图片的名称}
        \label{fig:campus}
  % flip
  \end{figure}
\end{frame}



\section{ 问题该如何解决}
\subsection{ CCB 评审}



\begin{frame}{ 最近的一次修正提交}
  不支持原子rmfb 的显卡都会走这个逻辑,虚拟机的umd的显示驱动vmwgfx要求fb
  要彻底删除。rmfb 的逻辑涉及的范围很广。而且这个逻辑的小修改自 2016 年以来一直
  没有新的改动,所以是没有大问题的。在公共部分加一些延时的trick 来修正 Oland 的显卡不是好办法。
  \begin{figure}[thpb]
        \centering
        \resizebox{0.7\linewidth}{!}{
          \includegraphics[width=2in]{figures/git.png}
        }
        % \includegraphics[scale=1.0]{figurefile}
        % \caption{ 此处是图片的名称}
        \label{fig:campus}
  % flip
  \end{figure}
\end{frame}




\begin{frame}{}
  切换用户过程中,宏观性能下降,闪屏的问题都是不支持 atomic\_remove\_fb 这个
  函数的遗留问题。这个问题是显卡的特性,经过一段时间的查询和思考是不能修改的。
  所以建议 CCB 评审关闭。

  \\~

  占老板还说过一个问题调用dpms的时候关crtc,渲染还没有结束,会导致花屏。这种问题
  其实也应该归为原子性的问题,如果是支持 drm\_atomic\_connector\_commit\_dpms 特性的显卡
  就不会出现问题。显卡在发展中,历史上出现的一些小的不合理的架构设计也在不断完善,因为
  硬件架构设计不支持的特性不能在内核当中解决。

\end{frame}


\section{Future Work}
\begin{frame}{Future Work}  %将来可做的方向
  \begin{itemize}
  \item<0-> 总结调式经验,加快分析速度,提升分析能力,多看显卡驱动的代码。
  \item<0-> 加深 drm 代码和 xorg 主要功能过程的的理解。
  \item<0-> 因为用户态和内核态都有涉及,增加用户态 mesa 的代码熟悉。
  \end{itemize}
\end{frame}

\begin{frame}{Thank you}
  \begin{center}
    \begin{minipage}{1\textwidth}
      \setbeamercolor{mybox}{fg=white, bg=black!50!blue}
      \begin{beamercolorbox}[wd=0.70\textwidth, rounded=true, shadow=true]{mybox}
        \LARGE \centering Thank you for listening!  %结束语
      \end{beamercolorbox}
    \end{minipage}
  \end{center}
\end{frame}

\begin{frame}{Q\&A}
  \begin{center}
    \begin{minipage}{1\textwidth}
      \setbeamercolor{mybox}{fg=white, bg=black!50!blue}
      \begin{beamercolorbox}[wd=0.70\textwidth, rounded=true, shadow=true]{mybox}
        \LARGE \centering  Questions?  %请求提问
      \end{beamercolorbox}
    \end{minipage}
  \end{center}
\end{frame}

% -----------------------------------------------------------------------------
\end{document}
% 文档结束
%%% Local Variables:
%%% mode: latex
%%% TeX-master: t
%%% End:
